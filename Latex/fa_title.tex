%% -!TEX root = AUTthesis.tex
% در این فایل، عنوان پایان‌نامه، مشخصات خود، متن تقدیمی‌، ستایش، سپاس‌گزاری و چکیده پایان‌نامه را به فارسی، وارد کنید.
% توجه داشته باشید که جدول حاوی مشخصات پروژه/پایان‌نامه/رساله و همچنین، مشخصات داخل آن، به طور خودکار، درج می‌شود.
%%%%%%%%%%%%%%%%%%%%%%%%%%%%%%%%%%%%
% دانشکده، آموزشکده و یا پژوهشکده  خود را وارد کنید
\faculty{دانشکده مهندسی کامپیوتر}
% گرایش و گروه آموزشی خود را وارد کنید
\department{}
% عنوان پایان‌نامه را وارد کنید
\fatitle{طبقه بندی ترافیک شبکه 
\\[.75 cm]
 با استفاده از الگوریتم‌های یادگیری ماشین}
% نام استاد(ان) راهنما را وارد کنید
\firstsupervisor{دکتر رضا صفابخش}
%\secondsupervisor{استاد راهنمای دوم}
% نام استاد(دان) مشاور را وارد کنید. چنانچه استاد مشاور ندارید، دستور پایین را غیرفعال کنید.
%\firstadvisor{نام کامل استاد مشاور}
%\secondadvisor{استاد مشاور دوم}
% نام نویسنده را وارد کنید
\name{ محمدمهدی}
% نام خانوادگی نویسنده را وارد کنید
\surname{ هجرتی}
%%%%%%%%%%%%%%%%%%%%%%%%%%%%%%%%%%
\thesisdate{اردیبهشت 1400}

% چکیده پایان‌نامه را وارد کنید
\fa-abstract{
امروزه با توجه به استفاده‌ی روزافزون از شبکه‌ی اینترنت در دنیا، افزایش سریع تعداد کاربران و ظهور
برنامه‌های کاربردی تحت شبکه، ترافیک اینترنت به شدت در حال افزایش است. در نتیجه شناسایی
برنامه‌ها در شبکه، به امر پیچیده‌ای تبدیل شده است. از طرفی طبقه‌بندی جریان‌ها نقش مهمی در
امنیت و مدیریت شبکه و به‌ویژه برای مقابله با حملات دارد.
در گذشته از روش‌های گوناگونی برای طبقه‌بندی ترافیک اینترنت از جمله روش‌های مبتنی بر درگاه، یا
بررسی پیلود بسته‌ها استفاده می‌شد. اما امروزه با توجه به مشکلات و محدودیت‌های موجود در 
روش‌های قبل مثل اختصاص دادن درگاه به صورت پویا، وجود داده های رمزگذاری شده و ... ناگزیر مجبور
به استفاده از روش‌های جدید مثل یادگیری ماشین شده‌ایم.
روش‌ها و الگوریتم‌های متعددی با استفاده از یادگیری ماشین برای طبقه‌بندی ترافیک شبکه پیشنهاد
شده‌است. هدف این پژوهش بررسی این روش‌ها و ارزیابی و مقایسه‌ی روش‌های پیشنهادی موجود
می‌باشد.
}


% کلمات کلیدی پایان‌نامه را وارد کنید
\keywords{ترافیک شبکه، یادگیری ماشین، طبقه‌بندی، دسته‌بندی}



\AUTtitle
%%%%%%%%%%%%%%%%%%%%%%%%%%%%%%%%%%
%\vspace*{7cm}
%\thispagestyle{empty}
%\begin{center}
%\includegraphics[height=5cm,width=12cm]{besm}
%\end{center}