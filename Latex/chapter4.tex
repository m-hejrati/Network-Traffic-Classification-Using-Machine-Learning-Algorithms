\chapter{ نتیجه گیری و پیشنهادها}
\section{نتیجه گیری}

شناسایی جریان جاری روی ترافیک اینترنت روی جنبه های مختلف شبکه مانند امنیت تأثیر زیادی دارد. همچنین تشخیص جریان های ترافیک باعث برنامه ریزی صحیح در قسمت های مختلف شبکه مانند تخصیص منابع، بهبود کیفیت خدمات سرویس و غیره می شود. بنابراین، با توجه به اهمیت شناسایی جریان های ترافیک اینترنت، در این پژوهش انواع روش های موجود برای طبقه بندی ترافیک که از گذشته تا کنون استفاده می‌شده است، مورد بررسی قرار گرفت و در مورد مزایا و معایب هرکدام بحث شد. در ادامه آزمایش های انجام شده برای شش مورد از الگوریتم های یادگیری ماشین برای طبقه بندی ترافیک شبکه با یکدیگر مقایسه شد که دیدیم الگوریتم \lr{C4.5} با دقت بیشتری نسبت به سایر الگوریتم ها این کار را انجام می دهد.

\section{پیشنهادها}
امروزه با گسترش علم یادگیری ماشین همچنان می توان امیدوار بود که با بهبود هر کدام از الگوریتم های موجود بتوان به دقت بسیار بیشتری نسبت به آنچه تا کنون بدست آمده برسیم. امید است با مطالعه ی بیشتر بر روی الگوریتم های یادگیری ماشین و ترکیب یا بهبود روش های موجود به دقت بالاتری در طبقه بندی ترافیک و درنتیجه امنیت و کیفیت بالاتری برای شبکه ی اینترنت برسیم.