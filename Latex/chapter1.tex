\thispagestyle{empty}
\chapter{مقدمه}

\section*{مقدمه}
امروزه طبقه بندی ترافیک شبکه به یک موضوع مهم در حوزه‌ی کامپیوتر تبدیل شده‌است. برای ارائه‌دهندگان خدمات اینترنت\LTRfootnote{Internet service providers}، آگاهی از برنامه‌های اجرا شده در شبکه یک امر حیاتی می‌باشد. طبقه‌بندی ترافیک شبکه اولین مرحله برای تجزیه و تحلیل و شناسایی انواع مختلف برنامه‌های شبکه است. با این روش ارائه‌دهندگان خدمات اینترنتی یا اپراتورهای شبکه می‌توانند عملکرد کلی یک شبکه را مدیریت کنند.
\\
روش های اولیه ی طبقه بندی ترافیک اینترنت مبتنی بر بازررسی بسته\LTRfootnote{packet} های جریان بودند. روش مبتنی بر شماره درگاه\LTRfootnote{port}، شماره درگاه در سرآیند\LTRfootnote{header}، بسته ها را با شماره های درگاه ثبت شده در مرجع شماره های اختصاص داده شده اینترنت مقایسه می کند. این روش برای جریان های با شماره درگاه پویا قابل اجرا نیست.\cite{iana}
\\
روش طبقه بندی مبتنی بر پیلود\LTRfootnote{payload}، تشخیص نوع برنامه را با پیدا کردن برخی از ویژگی های منحصر به فرد برنامه ها انجام می دهد. این روش روی بازرسی داده های کاربر تکیه دارد و درنتیجه، باعث نقص حریم خصوصی کاربر می شود.
\\
روش مبتنی بر رفتار میزبان، مستقل از بازرسی بسته های جریان، با نظارت بر همه جریان های ارسالی یا دریافتی روی میزبان های شبکه، می تواند ترافیک ایجاد شده توسط برنامه ها را طبقه بندی کند. این روش مبتنی بر این فرض است که میزبان در هر لحظه یک برنامه را اجرا می کند. که در واقعیت معمولا این طور نیست.
\\
امروزه متداول‌ترین تکنیک مورد استفاده، یادگیری ماشین\LTRfootnote{machine learning} است، که توسط بسیاری از محققان استفاده می‌شود و باعث بدست آمدن نتایج به مراتب دقیق‌تری از روش‌های پیشین شده است. تکنیک‌های یادگیری ماشین، با استفاده از مجموعه ویژگی های آماری جریان به طور خودکار الگوهای ساختاری موجود در انتقال داده‌های جریان را کشف می کنند. این روش می تواند مشکلاتی مانند شماره درگاه پویا، عدم حفظ حریم خصوصی کاربران و فرض عدم اجرای همزمان چند برنامه روی یک میزبان را رفع نماید.
هدف از این پژوهش بررسی این روش ها و ارزیابی و مقایسه‌ی روش‌های پیشنهادی موجود می‌باشد.
\\
در ادامه، در فصل دوم روش‌های مختلف موجود برای طبقه‌بندی ترافیک شبکه و مشکلات موجود بیان شده‌است. در فصل سوم مدل‌سازی و روش پیاده‌سازی روش‌های مبتنی بر یادگیری ماشین مورد بررسی قرار گرفته‌است. و در نهایت جمع‌بندی، نتیجه‌گیری و پیشنهادها در فصل پنجم ارائه شده‌ا‌‌ست.
